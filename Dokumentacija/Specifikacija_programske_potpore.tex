\chapter{Specifikacija programske potpore}
		
	\section{Funkcionalni zahtjevi}
			
			%\textbf{\textit{dio 1. revizije}}\\
			
			%\textit{Navesti \textbf{dionike} koji imaju \textbf{interes u ovom sustavu} ili  \textbf{su nositelji odgovornosti}. To su prije svega korisnici, ali i administratori sustava, naručitelji, razvojni tim.}\\
				
			%\textit{Navesti \textbf{aktore} koji izravno \textbf{koriste} ili \textbf{komuniciraju sa sustavom}. Oni mogu imati inicijatorsku ulogu, tj. započinju određene procese u sustavu ili samo sudioničku ulogu, tj. obavljaju određeni posao. Za svakog aktora navesti funkcionalne zahtjeve koji se na njega odnose.}\\
			
			
			\noindent \textbf{Dionici:}
			
			\begin{packed_enum}
				
				\item Korisnici
				\item[]\begin{packed_enum}
					\item Neregistrirani korisnici
					\item Klijenti
					\item Kulinarski entuzijasti
					\item Nutricionisti
				\end{packed_enum}
				\item Administratori
				
			\end{packed_enum}
			
			\noindent \textbf{Aktori i njihovi funkcionalni zahtjevi:}
			
			\begin{packed_enum}
				\item  \underbar{Neregistrirani korisnik (inicijator) može:}
				
				\begin{packed_enum}
					
					\item Pregledavati kuharice sortirane po novosti na prednjoj stranici
					\item Poslati zahtjev za registraciju:
					\begin{packed_enum}
						
						\item  Za klijenta s korisničkim imenom, lozinkom te imenom i prezimenom 
						\item  Za kulinarskog entuzijasta ili nutricionista sa dodatnom slikom, e-adresom i kratkom biografijom
				
					\end{packed_enum}
					\item  Pretraživati profile kulinarskih entuzijasta
					\item Pretraživati i anonimno komentirati kuharice
					
				\end{packed_enum}

				\item  \underbar{Klijent (inicijator) može:}
				
				\begin{packed_enum}
					
					\item Sve što i neregistrirani korisnik
					\item Odabrati dijetu koju prate
					\item Pretraživati recepte skeniranjem QR koda proizvoda
					\item Označiti konzumirane recepte
					
				\end{packed_enum}
				
				\item  \underbar{Kulinarski entuzijast (inicijator) može:}
				
				\begin{packed_enum}
					
					\item Sve što i klijent
					\item Kreirati kuharice, to jest tematski povezane skupove recepata
					
				\end{packed_enum}

				\item  \underbar{Nutricionist (inicijator) može:}
				
				\begin{packed_enum}
					
					\item Sve što i klijent
					\item Unositi informacije o proizvodima
					\item Kategorizirati proizvode
					\item Kreirati dijete
					
				\end{packed_enum}

				\item  \underbar{Administrator (inicijator) može:}
				
				\begin{packed_enum}
					
					\item Sve što i klijent, nutricionist i kulinarski entuzijast
					\item Odobriti prijave nutricionista i kulinarskih entuzijasta
					\item Pisati u i čitati iz baze podataka
					
				\end{packed_enum}

				\item  \underbar{Baza podataka (sudionik) može:}
	
				\begin{packed_enum}
					
					\item Brinuti se da je zadovoljen model podataka
					\item čuvati trenutno stanje korisnika, kuharica, dijeti i recepata
					
				\end{packed_enum}
			\end{packed_enum}
			
			\eject 

%			\begin{packed_enum}
%				\item  \underbar{Aktor 1 (inicijator) može:}
%				
%				\begin{packed_enum}
%					
%					\item funkcionalnost 1
%					\item funkcionalnost 2
%					\begin{packed_enum}
%						
%						\item  podfunkcionalnost 1 
%						\item  podfunkcionalnost 2
%				
%					\end{packed_enum}
%					\item  funkcionalnost 3
%					
%				\end{packed_enum}
%			
%				\item  \underbar{Aktor 2 (sudionik) može:}
%				
%				\begin{packed_enum}
%					
%					\item funkcionalnost 1
%					\item funkcionalnost 2
%					
%				\end{packed_enum}
%			\end{packed_enum}
%			
%			\eject 

				
			\subsection{Obrasci uporabe}
%				
%				\textbf{\textit{dio 1. revizije}}
%				
				\subsubsection{Opis obrazaca uporabe}
					\noindent \underbar{\textbf{UC1 - Pregled novih kuharica}}
					\begin{packed_item}
	
						\item \textbf{Glavni sudionik: }Neregistrirani korisnik, Klijent, Kulinarski entuzijast, Administrator
						\item  \textbf{Cilj:} Pregledavanje kuharica
						\item  \textbf{Sudionici:} Baza podataka
						\item  \textbf{Preduvjet:} -
						\item  \textbf{Opis osnovnog tijeka:} 
						
						\item[] \begin{packed_enum}
	
							\item Nove kuharice su prikazane prilikom učitavanja aplikacije
							\item Korisnik odabire jednu od navedenih kuharica
							\item Prikazuje se lista recepata unutar navedene kuharice
						\end{packed_enum}
						
						\item  \textbf{Opis mogućih odstupanja:}
						
						\item[] \begin{packed_item}
	
							\item[2.a] Nema kuharica u bazi podataka
							\item[] \begin{packed_enum}
								
								\item Sustav ispisuje poruku da nema kuharica u bazi podataka
								
							\end{packed_enum}

							
						\end{packed_item}
					\end{packed_item}



					\noindent \underbar{\textbf{UC2 - Registracija klijenta}}
					\begin{packed_item}
	
						\item \textbf{Glavni sudionik: }Neregistrirani korisnik
						\item  \textbf{Cilj:} Stvaranje korisničkog računa s statusom klijenta
						\item  \textbf{Sudionici:} Baza podataka
						\item  \textbf{Preduvjet:} -
						\item  \textbf{Opis osnovnog tijeka:} 
						
						\item[] \begin{packed_enum}
	
							\item pritiskom na gumb "Sign up" otvara se sučelje za registraciju
							\item Korisnik unosi podatke o korisničkom imenu, lozinki, imenu i prezimenu
							\item Korisnik prima obavijest o uspješnoj registraciji
						\end{packed_enum}
						
						\item  \textbf{Opis mogućih odstupanja:}
						
						\item[] \begin{packed_item}
	
							\item[2.a] Odabir već zauzetog korisničkog imena, unos korisničkog podatka u nedozvoljenom formatu
							\item[] \begin{packed_enum}
								
								\item Sustav obavještava korisnika o neuspjeloj registraciji i vraća ga u sučelje za registraciju
								\item Korisnik mijenja potrebne podatke i završava registraciju ili odustaje od registracije
								
							\end{packed_enum}

							
						\end{packed_item}
					\end{packed_item}



					\noindent \underbar{\textbf{UC3 - Registracija kulinarskog entuzijasta ili nutricionista}}
					\begin{packed_item}
	
						\item \textbf{Glavni sudionik: }Neregistrirani korisnik
						\item  \textbf{Cilj:} Stvaranje korisničkog računa s statusom kulinarskog entuzijasta ili nutricionista
						\item  \textbf{Sudionici:} Baza podataka, Administrator
						\item  \textbf{Preduvjet:} -
						\item  \textbf{Opis osnovnog tijeka:} 
						
						\item[] \begin{packed_enum}
	
							\item pritiskom na gumb "Sign up" otvara se sučelje za registraciju
							\item Korisnik unosi podatke o korisničkom imenu, lozinki, imenu i prezimenu, e-adresom, kratkom biografijom i slikom
							\item Korisnik prima obavijest o uspješnoj registraciji nakon odobrenja administratora
						\end{packed_enum}
						
						\item  \textbf{Opis mogućih odstupanja:}
						
						\item[] \begin{packed_item}
	
							\item[2.a] Odabir već zauzetog korisničkog imena i/ili e-maila, unos korisničkog podatka u nedozvoljenom formatu
							\item[] \begin{packed_enum}
								
								\item Sustav obavještava korisnika o neuspjeloj registraciji i vraća ga u sučelje za registraciju
								\item Korisnik mijenja potrebne podatke i završava registraciju ili odustaje od registracije
								
							\end{packed_enum}
							\item[2.b] Administrator ne odobrava registraciju
							\item[] \begin{packed_enum}
								
								\item Sustav obavještava korisnika o neuspjeloj registraciji i vraća ga u sučelje za registraciju
								\item Korisnik mijenja potrebne podatke i završava registraciju ili odustaje od registracije
								
							\end{packed_enum}
							
						\end{packed_item}
					\end{packed_item}



					\noindent \underbar{\textbf{UC4 - Pregled profila kulinarskih entuzijasta}}
					\begin{packed_item}
	
						\item \textbf{Glavni sudionik: }Neregistrirani korisnik, Klijent, Kulinarski entuzijast, Nutricionist, Administrator
						\item  \textbf{Cilj:} Pregled profila kulinarskih entuzijasta
						\item  \textbf{Sudionici:} Baza podataka
						\item  \textbf{Preduvjet:} -
						\item  \textbf{Opis osnovnog tijeka:} 
						
						\item[] \begin{packed_enum}
	
							\item korisnik pritišće ime kulinarskog entuzijasta unutar kuharice
							\item Otvara se sučelje s podatcima o kulinarskom entuzijastu
						\end{packed_enum}
						

					\end{packed_item}



					\noindent \underbar{\textbf{UC5 - Pretraživanje profila kulinarskih entuzijasta i kuharica}}
					\begin{packed_item}
	
						\item \textbf{Glavni sudionik: }Neregistrirani korisnik, Klijent, Kulinarski entuzijast, Nutricionist, Administrator
						\item  \textbf{Cilj:} Pregled profila i kuharica koje sadrže ključnu riječ
						\item  \textbf{Sudionici:} Baza podataka
						\item  \textbf{Preduvjet:} -
						\item  \textbf{Opis osnovnog tijeka:} 
						
						\item[] \begin{packed_enum}
	
							\item korisnik u tražilicu upisuje jednu ili više ključnih riječi 
							\item Korisnik pritišće gumb za pretraživanje
							\item Otvara se sučelje s listom profila i kuharica koje sadrže ključne riječi
						\end{packed_enum}
						
						\item  \textbf{Opis mogućih odstupanja:}
						
						\item[] \begin{packed_item}
	
							\item[2.a] Niti jedan profil ili kuharica ne sadrži ključnu riječ
							\item[] \begin{packed_enum}
								
								\item Sustav obavještava korisnika o neuspjelom pretraživanju 
								\item Klijent se vrati u sučelje gdje je bio prije pretraživanja
								
							\end{packed_enum}

							\item[2.b] Nije upisana ključna riječ prije pretraživanja
							\item[] \begin{packed_enum}
								
								\item Sustav obavještava korisnika o neuspjelom pretraživanju 
								\item Klijent se vrati u sučelje gdje je bio prije pretraživanja
								
							\end{packed_enum}							
						\end{packed_item}
					\end{packed_item}



					\noindent \underbar{\textbf{UC6 - Anonimno komentiranje kuharica}}
					\begin{packed_item}
	
						\item \textbf{Glavni sudionik: }Neregistrirani korisnik
						\item  \textbf{Cilj:} Komentiranje kuharica
						\item  \textbf{Sudionici:} Baza podataka
						\item  \textbf{Preduvjet:} -
						\item  \textbf{Opis osnovnog tijeka:} 
						
						\item[] \begin{packed_enum}
	
							\item korisnik unutar kuharice pritišće gumb "Komentiraj"
							\item Otvara se sučelje gdje korisnik upisuje tekst
							\item Pritiskom na gumb "Spremi", komentar se sprema u bazu podataka i vidljiv je unutar kuharice
						\end{packed_enum}
						
						\item  \textbf{Opis mogućih odstupanja:}
						
						\item[] \begin{packed_item}
	
							\item[2.a] Pokušaj spremanja praznog komentara
							\item[] \begin{packed_enum}
								
								\item Sustav obavještava korisnika o neuspjelom pokušaju spremanja komentara 
								\item Klijent ispuni polje za komentar ili odustane
								
							\end{packed_enum}

						\end{packed_item}
					\end{packed_item}



					\noindent \underbar{\textbf{UC7 - Prijava u sustav}}
					\begin{packed_item}
	
						\item \textbf{Glavni sudionik: } Klijent, Kulinarski entuzijast, Nutricionist, Administrator
						\item  \textbf{Cilj:} Prijava u sustav kao registrirani korisnik
						\item  \textbf{Sudionici:} Baza podataka
						\item  \textbf{Preduvjet:} Registracija profila
						\item  \textbf{Opis osnovnog tijeka:} 
						
						\item[] \begin{packed_enum}
	
							\item Korisnik odabire opciju "Prijavi se"
							\item Korisnik ispunjava potrebne podatke
							\item Korisnik je prijavljen i vraća se na početnu stranicu
						\end{packed_enum}
						
						\item  \textbf{Opis mogućih odstupanja:}
						
						\item[] \begin{packed_item}
	
							\item[2.a] Neispravni unos podataka
							\item[] \begin{packed_enum}
								
								\item Sustav obavještava korisnika o neispravnim podatcima i traži ponovni unos podataka
								
							\end{packed_enum}

						\end{packed_item}
					\end{packed_item}


					\noindent \underbar{\textbf{UC8 - Odabir dijete koje će korisnik pratiti}}
					\begin{packed_item}
	
						\item \textbf{Glavni sudionik: }Klijent, Kulinarski entuzijast, Nutricionist, Administrator 
						\item  \textbf{Cilj:} Odabir dijete
						\item  \textbf{Sudionici:} Baza podataka
						\item  \textbf{Preduvjet:} -
						\item  \textbf{Opis osnovnog tijeka:} 
						
						\item[] \begin{packed_enum}
	
							\item korisnik odabire gumb "Odabir dijete"
							\item Otvara se sučelje gdje korisnik odabire jednu od navedenih dijeta
							\item Pritiskom na gumb "Spremi", odabrana dijeta se sprema i vidljiva je u receptima koje korisnik pregledava ili filtrira recepte koji zadovoljavaju dijetu - potrebno je odlućuti kaj će dijeta radit
						\end{packed_enum}
						
						\item  \textbf{Opis mogućih odstupanja:}
						
						\item[] \begin{packed_item}
	
							\item[2.a] Pokušaj spremanja bez odabira dijete
							\item[] \begin{packed_enum}
								
								\item Sustav obavještava korisnika o neuspjelom pokušaju spremanja dijete
								\item Klijent ispuni polje za dijetu ili odustane
								
							\end{packed_enum}							
							\item[2.b] Nema dijeta u sustavu
							\item[] \begin{packed_enum}
								
								\item Sustav obavještava korisnika o nedostatku dijeta
								
							\end{packed_enum}

						\end{packed_item}
					\end{packed_item}





					\noindent \underbar{\textbf{UC9 - Pretraživanje unosom QR koda}}
					\begin{packed_item}
	
						\item \textbf{Glavni sudionik: }Klijent, Kulinarski entuzijast, Nutricionist, Administrator 
						\item  \textbf{Cilj:} Na temelju koda na proizvodu vidjeti recepte u kojima se on nalazi
						\item  \textbf{Sudionici:} Baza podataka
						\item  \textbf{Preduvjet:} -
						\item  \textbf{Opis osnovnog tijeka:} 
						
						\item[] \begin{packed_enum}
	
						\item Korisnik bira opciju "Skeniraj kod proizvoda"
						\item Korisnik je poslan na stranicu s opcijom postavljanja slike na poslužitelj
						\item Korisnik postavlja sliku i čeka njenu obradu
						\item Korisnik dobiva popis recepata koji sadrže skenirani proizvod
						\end{packed_enum}
						
						\item  \textbf{Opis mogućih odstupanja:}
						
						\item[] \begin{packed_item}
	
							\item[2.a] Slika nije u podržanom formatu
							\item[] \begin{packed_enum}
								
								\item Sustav ispisuje poruku da slika nije u podržanom formatu i traži se ponovni unos slike
								
							\end{packed_enum}							
							\item[2.b] Kod u slici nije pronađen
							\item[] \begin{packed_enum}
								
								\item Sustav ispisuje poruku da kod na slici nije prepoznat i traži se ponovni unos slike
								
							\end{packed_enum}

						\end{packed_item}
					\end{packed_item}



					\noindent \underbar{\textbf{UC10 - Kreiranje kuharice}}
					\begin{packed_item}
	
						\item \textbf{Glavni sudionik: } Kulinarski entuzijast, Administrator 
						\item  \textbf{Cilj:} Kreirati kuharicu koja sadrži različite recepte
						\item  \textbf{Sudionici:} Baza podataka
						\item  \textbf{Preduvjet:} -
						\item  \textbf{Opis osnovnog tijeka:} 
						
						\item[] \begin{packed_enum}
	
						\item Korisnik bira opciju "Kreiraj kuharicu" nakon čega se otvara sučelje za kreiranje kuharice
						\item Korisnik odabire gumb "Dodaj recept" i odabire recept koji će dodati u kuharicu				
						\item Korisnik odabire gumb "Obriši recept" i odabire recept koji će se maknuti iz kuharice
						\item Korisnik odabire gumb "Spremi kuharicu" čime se kuharica sprema u bazu podataka
						\end{packed_enum}
						
						\item  \textbf{Opis mogućih odstupanja:}
						
						\item[] \begin{packed_item}
	
							\item[2.a] Spremanje prazne kuharice
							\item[] \begin{packed_enum}
								
								\item Sustav ispisuje poruku da nema recepata u kuharici
								\item Korisnik dodaje recepte kuharicu ili odustaje
								
							\end{packed_enum}							


						\end{packed_item}
					\end{packed_item}


					\noindent \underbar{\textbf{UC11 - Uređivanje kuharice}}
					\begin{packed_item}
	
						\item \textbf{Glavni sudionik: } Kulinarski entuzijast, Administrator 
						\item  \textbf{Cilj:} Uređivanje jedne od vlastitih kuharica
						\item  \textbf{Sudionici:} Baza podataka
						\item  \textbf{Preduvjet:} -
						\item  \textbf{Opis osnovnog tijeka:} 
						
						\item[] \begin{packed_enum}
	
						\item Korisnik bira opciju "Uredi kuharicu" nakon čega se otvara sučelje za uređivanje kuharice
						\item Korisnik odabire gumb "Dodaj recept" i odabire recept koji će dodati u kuharicu				
						\item Korisnik odabire gumb "Obriši recept" i odabire recept koji će se maknuti iz kuharice
						\item Korisnik odabire gumb "Spremi kuharicu" čime se kuharica sprema u bazu podataka
						\end{packed_enum}
						
						\item  \textbf{Opis mogućih odstupanja:}
						
						\item[] \begin{packed_item}
	
							\item[2.a] Spremanje prazne kuharice
							\item[] \begin{packed_enum}
								
								\item Sustav ispisuje poruku da nema recepata u kuharici
								\item Korisnik dodaje recepte kuharicu ili odustaje
								
							\end{packed_enum}							


						\end{packed_item}
					\end{packed_item}



					\noindent \underbar{\textbf{UC12 - Kreiranje recepata}}
					\begin{packed_item}
	
						\item \textbf{Glavni sudionik: } Kulinarski entuzijast, Administrator 
						\item  \textbf{Cilj:} Kreirati kuharicu koja sadrži različite recepte
						\item  \textbf{Sudionici:} Baza podataka
						\item  \textbf{Preduvjet:} -
						\item  \textbf{Opis osnovnog tijeka:} 
						
						\item[] \begin{packed_enum}
	
						\item Korisnik bira opciju "Kreiraj recept" nakon čega se otvara sučelje za kreiranje recepata
						\item Korisnik odabire sastojke koji ulaze u recept i upisuje način pripreme jela
						\item Korisnik odabire gumb "Spremi recept" čime se recept sprema u bazu podataka
						\end{packed_enum}
						
						\item  \textbf{Opis mogućih odstupanja:}
						
						\item[] \begin{packed_item}
	
							\item[2.a] Spremanje praznog recepta
							\item[] \begin{packed_enum}
								
								\item Sustav ispisuje poruku da nema sastojaka ili opisa pripreme
								\item Korisnik dodaje potrebne informacije ili odustaje
								
							\end{packed_enum}							


						\end{packed_item}
					\end{packed_item}


					\noindent \underbar{\textbf{UC13 - Uređivanje recepata}}
					\begin{packed_item}
	
						\item \textbf{Glavni sudionik: } Kulinarski entuzijast, Administrator 
						\item  \textbf{Cilj:} Uređivanje jednog od vlastitih recepata
						\item  \textbf{Sudionici:} Baza podataka
						\item  \textbf{Preduvjet:} -
						\item  \textbf{Opis osnovnog tijeka:} 
						
						\item[] \begin{packed_enum}
	
						\item Korisnik bira opciju "Uredi recept" nakon čega se otvara sučelje za uređivanje recepata
						\item Korisnik mijenja sastojke koji ulaze u recept i mijenja opis pripreme
						\item Korisnik odabire gumb "Spremi recept" čime se recept sprema u bazu podataka
						\end{packed_enum}
						
						\item  \textbf{Opis mogućih odstupanja:}
						
						\item[] \begin{packed_item}
	
							\item[2.a] Spremanje praznog recepta
							\item[] \begin{packed_enum}
								
								\item Sustav ispisuje poruku da nema sastojaka ili opisa pripreme
								\item Korisnik dodaje potrebne informacije ili odustaje
								
							\end{packed_enum}						


						\end{packed_item}
					\end{packed_item}

                    \noindent \underbar{\textbf{UC14 - Unošenje proizvoda}}
                    \begin{packed_item}
    
                        \item \textbf{Glavni sudionik: }Nutricionist, Administrator
                        \item  \textbf{Cilj:} Unošenje novog proizvoda zajedno sa relevantnim atributima
                        \item  \textbf{Sudionici:} Baza podataka
                        \item  \textbf{Preduvjet:} Nema navedenog proizvoda u bazi podataka
                        \item  \textbf{Opis osnovnog tijeka:} 
                        
                        \item[] \begin{packed_enum}
    
                            \item Korisnik pritisne gumb "Dodaj proizvod"
                            \item Prikazuje se sučelje u koje korisnik upisuje potrebne podatke
                            \item Korisnik pritisne gumb "Spremi" kojim sprema proizvod zajedno sa vezanim informacijama u bazu podataka
                        \end{packed_enum}
                        
                        \item  \textbf{Opis mogućih odstupanja:}
                        
                        \item[] \begin{packed_item}
    
                            \item[2.a] Proizvod tog imena već postoji u bazi podataka
                            \item[] \begin{packed_enum}
                                
                                \item Sustav ispisuje poruku da je proizvod istog imena već u bazi podataka
                                \item Korisnik mijenja naziv proizvoda ili odustane
                                
                            \end{packed_enum}

                            \item[2.b] Unos informacija u nedozvoljenom formatu
                            \item[] \begin{packed_enum}
                                
                                \item Sustav ispisuje poruku s dojavom greške
                                \item Korisnik mijenja informacije ili odustane
                                
                            \end{packed_enum}

                        \end{packed_item}
                    \end{packed_item}

                    \noindent \underbar{\textbf{UC15 - Promjena informacija o proizvodima}}
                    \begin{packed_item}
    
                        \item \textbf{Glavni sudionik: }Nutricionist, Administrator
                        \item  \textbf{Cilj:} Promjena atributa postojećeg proizvoda
                        \item  \textbf{Sudionici:} Baza podataka
                        \item  \textbf{Preduvjet:} Postoji navedeni proizvod u bazi podataka
                        \item  \textbf{Opis osnovnog tijeka:} 
                        
                        \item[] \begin{packed_enum}
    
                            \item Korisnik pritisne gumb "Promijeni proizvod"
                            \item Prikazuje se sučelje u koje korisnik mijenja podatke proizvoda
                            \item Korisnik pritisne gumb "Spremi" kojim sprema proizvod zajedno sa promjenjenim informacijama u bazu podataka
                        \end{packed_enum}
                        
                        \item  \textbf{Opis mogućih odstupanja:}
                        
                        \item[] \begin{packed_item}
    
                            \item[2.a] Unos informacija u nedozvoljenom formatu
                            \item[] \begin{packed_enum}
                                
                                \item Sustav ispisuje poruku s dojavom greške
                                \item Korisnik mijenja informacije ili odustane
                                
                            \end{packed_enum}

                            
                        \end{packed_item}
                    \end{packed_item}

                    \noindent \underbar{\textbf{UC16 - Brisanje proizvoda}}
                    \begin{packed_item}
    
                        \item \textbf{Glavni sudionik: }Nutricionist, Administrator
                        \item  \textbf{Cilj:} Brisanje postojećeg proizvoda
                        \item  \textbf{Sudionici:} Baza podataka
                        \item  \textbf{Preduvjet:} Postoji navedeni proizvod u bazi podataka
                        \item  \textbf{Opis osnovnog tijeka:} 
                        
                        \item[] \begin{packed_enum}
    
                            \item Korisnik pritisne gumb "Briši proizvod"
                            \item Prikazuje se sučelje u kojemu se daje korisniku opcija "Briši" ili "Odustani"
                            \item Korisnik pritisne gumb "Briši" čime se briše proizvod iz baze podataka ili pritisne gumb "Odustani" čime se zaustavi brisanje proizvoda
                        \end{packed_enum}
                        
                    \end{packed_item}

                    \noindent \underbar{\textbf{UC17 - Kategorizirati proizvode}}
                    \begin{packed_item}
    
                        \item \textbf{Glavni sudionik: }Nutricionist, Administrator
                        \item  \textbf{Cilj:} Dodavanje kategorije postojećem proizvodu
                        \item  \textbf{Sudionici:} Baza podataka
                        \item  \textbf{Preduvjet:} Postoji navedeni proizvod u bazi podataka
                        \item  \textbf{Opis osnovnog tijeka:} 
                        
                        \item[] \begin{packed_enum}
    
                            \item Korisnik pritisne gumb "Kategoriziraj proizvod"
                            \item Prikazuje se sučelje u koje korisnik dodaje ili mijenja kategoriju proizvoda
                            \item Korisnik pritisne gumb "Spremi" kojim sprema kategoriju proizvoda u bazu podataka
                        \end{packed_enum}
                        
                        \item  \textbf{Opis mogućih odstupanja:}
                        
                        \item[] \begin{packed_item}
    
                            \item[2.a] Unos informacija u nedozvoljenom formatu
                            \item[] \begin{packed_enum}
                                
                                \item Sustav ispisuje poruku s dojavom greške
                                \item Korisnik mijenja informacije ili odustane
                                
                            \end{packed_enum}

                            
                        \end{packed_item}
                    \end{packed_item}

                    \noindent \underbar{\textbf{UC18 - Dodavanje kategorija}}
                    \begin{packed_item}
    
                        \item \textbf{Glavni sudionik: }Nutricionist, Administrator
                        \item  \textbf{Cilj:} Dodavanje nove kategorije
                        \item  \textbf{Sudionici:} Baza podataka
                        \item  \textbf{Preduvjet:} Nema navedene kategorije u bazi podataka
                        \item  \textbf{Opis osnovnog tijeka:} 
                        
                        \item[] \begin{packed_enum}
    
                            \item Korisnik pritisne gumb "Dodaj kategoriju"
                            \item Prikazuje se sučelje u koje korisnik dodaje kategoriju
                            \item Korisnik pritisne gumb "Spremi" kojim sprema kategoriju proizvoda u bazu podataka
                        \end{packed_enum}
                        
                        \item  \textbf{Opis mogućih odstupanja:}
                        
                        \item[] \begin{packed_item}
    
                            \item[2.a] Kategorija tog imena već postoji u bazi podataka
                            \item[] \begin{packed_enum}
                                
                                \item Sustav ispisuje poruku da je kategorija istog imena već u bazi podataka
                                \item Korisnik mijenja naziv kategorije ili odustane
                                
                            \end{packed_enum}

                            \item[2.b] Unos informacija u nedozvoljenom formatu
                            \item[] \begin{packed_enum}
                                
                                \item Sustav ispisuje poruku s dojavom greške
                                \item Korisnik mijenja informacije ili odustane
                                
                            \end{packed_enum}

                            
                        \end{packed_item}
                    \end{packed_item}

                    \noindent \underbar{\textbf{UC19 - Promjena kategorije}}
                    \begin{packed_item}
    
                        \item \textbf{Glavni sudionik: }Nutricionist, Administrator
                        \item  \textbf{Cilj:} Promjena atributa kategorije
                        \item  \textbf{Sudionici:} Baza podataka
                        \item  \textbf{Preduvjet:} Postoji navedena kategorija u bazi podataka
                        \item  \textbf{Opis osnovnog tijeka:} 
                        
                        \item[] \begin{packed_enum}
    
                            \item Korisnik pritisne gumb "Promijeni kategoriju"
                            \item Prikazuje se sučelje gdje korisnik mijenja atribute kategorije
                            \item Korisnik pritisne gumb "Spremi" kojim sprema promjene u bazu podataka
                        \end{packed_enum}
                        
                        \item  \textbf{Opis mogućih odstupanja:}
                        
                        \item[] \begin{packed_item}

                            \item[2.a] Unos informacija u nedozvoljenom formatu
                            \item[] \begin{packed_enum}
                                
                                \item Sustav ispisuje poruku s dojavom greške
                                \item Korisnik mijenja informacije ili odustane
                                
                            \end{packed_enum}

                            
                        \end{packed_item}
                    \end{packed_item}

                    \noindent \underbar{\textbf{UC20 - Brisanje kategorija}}
                    \begin{packed_item}
    
                        \item \textbf{Glavni sudionik: }Nutricionist, Administrator
                        \item  \textbf{Cilj:} Brisanje postojeće kategorije
                        \item  \textbf{Sudionici:} Baza podataka
                        \item  \textbf{Preduvjet:} Postoji navedena kategorije u bazi podataka
                        \item  \textbf{Opis osnovnog tijeka:} 
                        
                        \item[] \begin{packed_enum}
    
                            \item Korisnik pritisne gumb "Briši kategoriju"
                            \item Prikazuje se sučelje u kojemu se daje korisniku opcija "Briši" ili "Odustani"
                            \item Korisnik pritisne gumb "Briši" čime se briše kategorija iz baze podataka ili pritisne gumb "Odustani" čime se zaustavi brisanje kategorije
                        \end{packed_enum}
                        
                    \end{packed_item}


                    \noindent \underbar{\textbf{UC21 - Dodavanje dijeta}}
                    \begin{packed_item}
    
                        \item \textbf{Glavni sudionik: }Nutricionist, Administrator
                        \item  \textbf{Cilj:} Dodavanje nove dijete
                        \item  \textbf{Sudionici:} Baza podataka
                        \item  \textbf{Preduvjet:} Nema navedene dijete u bazi podataka
                        \item  \textbf{Opis osnovnog tijeka:} 
                        
                        \item[] \begin{packed_enum}
    
                            \item Korisnik pritisne gumb "Dodaj dijetu"
                            \item Prikazuje se sučelje u koje korisnik dodaje dijetu
                            \item Korisnik pritisne gumb "Spremi" kojim sprema dijetu u bazu podataka
                        \end{packed_enum}
                        
                        \item  \textbf{Opis mogućih odstupanja:}
                        
                        \item[] \begin{packed_item}
    
                            \item[2.a] Dijeta tog imena već postoji u bazi podataka
                            \item[] \begin{packed_enum}
                                
                                \item Sustav ispisuje poruku da je dijeta istog imena već u bazi podataka
                                \item Korisnik mijenja naziv dijete ili odustane
                                
                            \end{packed_enum}

                            \item[2.b] Unos informacija u nedozvoljenom formatu
                            \item[] \begin{packed_enum}
                                
                                \item Sustav ispisuje poruku s dojavom greške
                                \item Korisnik mijenja informacije ili odustane
                                
                            \end{packed_enum}

                            
                        \end{packed_item}
                    \end{packed_item}

                    \noindent \underbar{\textbf{UC22 - Promjena dijeta}}
                    \begin{packed_item}
    
                        \item \textbf{Glavni sudionik: }Nutricionist, Administrator
                        \item  \textbf{Cilj:} Promjena atributa i stavki dijete
                        \item  \textbf{Sudionici:} Baza podataka
                        \item  \textbf{Preduvjet:} Postoji navedena kategorija u bazi podataka
                        \item  \textbf{Opis osnovnog tijeka:} 
                        
                        \item[] \begin{packed_enum}
    
                            \item Korisnik pritisne gumb "Promijeni dijetu"
                            \item Prikazuje se sučelje gdje korisnik mijenja atribute i stavke dijete
                            \item Korisnik pritisne gumb "Spremi" kojim sprema promjene u bazu podataka
                        \end{packed_enum}
                        
                        \item  \textbf{Opis mogućih odstupanja:}
                        
                        \item[] \begin{packed_item}

                            \item[2.a] Unos informacija u nedozvoljenom formatu
                            \item[] \begin{packed_enum}
                                
                                \item Sustav ispisuje poruku s dojavom greške
                                \item Korisnik mijenja informacije ili odustane
                                
                            \end{packed_enum}

                            
                        \end{packed_item}
                    \end{packed_item}

                    \noindent \underbar{\textbf{UC23 - Brisanje dijeta}}
                    \begin{packed_item}
    
                        \item \textbf{Glavni sudionik: }Nutricionist, Administrator
                        \item  \textbf{Cilj:} Brisanje postojeće dijete
                        \item  \textbf{Sudionici:} Baza podataka
                        \item  \textbf{Preduvjet:} Postoji navedena dijeta u bazi podataka
                        \item  \textbf{Opis osnovnog tijeka:} 
                        
                        \item[] \begin{packed_enum}
    
                            \item Korisnik pritisne gumb "Briši dijetu"
                            \item Prikazuje se sučelje u kojemu se daje korisniku opcija "Briši" ili "Odustani"
                            \item Korisnik pritisne gumb "Briši" čime se briše dijeta iz baze podataka ili pritisne gumb "Odustani" čime se zaustavi brisanje dijete
                        \end{packed_enum}
                        
                    \end{packed_item}

                    \noindent \underbar{\textbf{UC24 - Prihvaćanje ili odbijanje prijave nutricionista i kulinarskih entuzijasta}}
                    \begin{packed_item}
    
                        \item \textbf{Glavni sudionik: }Administrator
                        \item  \textbf{Cilj:} Odobriti ili poništiti zahtjev za ulogu nutricionista ili kulinarskog entuzijasta korisniku
                        \item  \textbf{Sudionici:} Baza podataka
                        \item  \textbf{Preduvjet:} Postoji prijava za ulogu nutricionista ili kulinarskog entuzijasta
                        \item  \textbf{Opis osnovnog tijeka:} 
                        
                        \item[] \begin{packed_enum}
    
                            \item Korisnik pritisne gumb "Provjeri prijavu"
                            \item Prikazuje se sučelje u kojemu se daje korisniku opcija "Prihvati" ili "Odbij"
                            \item Korisnik pritisne gumb "Prihvati" čime se prihvaća prijava korisnika za ulogu u bazi podataka ili pritisne gumb "Odbij" čime se odbija prijava korisnika za traženu ulogu
                        \end{packed_enum}
                        
                    \end{packed_item}  					



			\noindent \underbar{\textbf{UC25 - Komentiranje kuharica}}
					\begin{packed_item}
	
						\item \textbf{Glavni sudionik: }Klijent, Kulinarski entuzijast, Nutricionist, Administrator 
						\item  \textbf{Cilj:} Komentiranje kuharica
						\item  \textbf{Sudionici:} Baza podataka
						\item  \textbf{Preduvjet:} -
						\item  \textbf{Opis osnovnog tijeka:} 
						
						\item[] \begin{packed_enum}
	
							\item korisnik unutar kuharice pritišće gumb "Komentiraj"
							\item Otvara se sučelje gdje korisnik upisuje tekst
							\item Pritiskom na gumb "Spremi", komentar se sprema u bazu podataka i vidljiv je unutar kuharice
						\end{packed_enum}
						
						\item  \textbf{Opis mogućih odstupanja:}
						
						\item[] \begin{packed_item}
	
							\item[2.a] Pokušaj spremanja praznog komentara
							\item[] \begin{packed_enum}
								
								\item Sustav obavještava korisnika o neuspjelom pokušaju spremanja komentara 
								\item Klijent ispuni polje za komentar ili odustane
								
							\end{packed_enum}

						\end{packed_item}
					\end{packed_item}




%%
%                   \noindent \underbar{\textbf{UC9 - Pretraživanje unosom QR koda}}
%                    \begin{packed_item}
%
%                        \item \textbf{Glavni sudionik: }Klijent
%                        \item  \textbf{Cilj:}Na temelju koda na proizvodu vidjeti recepte u kojima se on nalazi
%                        \item  \textbf{Sudionici:} Baza podataka
%                        \item  \textbf{Preduvjet:} Klijent je prijavljen
%                        \item  \textbf{Opis osnovnog tijeka:}
%
%                        \item[] \begin{packed_enum}
%
% 						\item Korisnik bira opciju "Skeniraj kod proizvoda"
%						\item Korisnik je poslan na stranicu s opcijom postavljanja slike na poslužitelj
%						\item Korisnik postavlja sliku i čeka njenu obradu
%						\item Korisnik dobiva popis recepata koji sadrže skenirani proizvod
%                         \end{packed_enum}
%					                        
%
%                         \item  \textbf{Opis mogućih odstupanja:}
%
%                         \item[] \begin{packed_item}
%
%                       	\item[3.a] Slika nije u podržanom formatu
%                             \item[] \begin{packed_enum}
%
%                                 \item Sustav ispisuje poruku da slika nije u podržanom formatu i traži se ponovni unos slike
%
%                             \end{packed_enum}
%					\item[3.b] Kod u slici nije pronađen
%					\item[] \begin{packed_enum}
%
%                                \item Sustav ispisuje poruku da kod na slici nije prepoznat i traži se ponovni unos slike
%
%                         \end{packed_enum}
%                    \end{packed_item}

%					\textit{Funkcionalne zahtjeve razraditi u obliku obrazaca uporabe. Svaki obrazac je potrebno razraditi prema donjem predlošku. Ukoliko u nekom koraku može doći do odstupanja, potrebno je to odstupanje opisati i po mogućnosti ponuditi rješenje kojim bi se tijek obrasca vratio na osnovni tijek.}\\
%					
%
%					\noindent \underbar{\textbf{UC$<$broj obrasca$>$ -$<$ime obrasca$>$}}
%					\begin{packed_item}
%	
%						\item \textbf{Glavni sudionik: }$<$sudionik$>$
%						\item  \textbf{Cilj:} $<$cilj$>$
%						\item  \textbf{Sudionici:} $<$sudionici$>$
%						\item  \textbf{Preduvjet:} $<$preduvjet$>$
%						\item  \textbf{Opis osnovnog tijeka:}
%						
%						\item[] \begin{packed_enum}
%	
%							\item $<$opis korak jedan$>$
%							\item $<$opis korak dva$>$
%							\item $<$opis korak tri$>$
%							\item $<$opis korak četiri$>$
%							\item $<$opis korak pet$>$
%						\end{packed_enum}
%						
%						\item  \textbf{Opis mogućih odstupanja:}
%						
%						\item[] \begin{packed_item}
%	
%							\item[2.a] $<$opis mogućeg scenarija odstupanja u koraku 2$>$
%							\item[] \begin{packed_enum}
%								
%								\item $<$opis rješenja mogućeg scenarija korak 1$>$
%								\item $<$opis rješenja mogućeg scenarija korak 2$>$
%								
%							\end{packed_enum}
%							\item[2.b] $<$opis mogućeg scenarija odstupanja u koraku 2$>$
%							\item[3.a] $<$opis mogućeg scenarija odstupanja  u koraku 3$>$
%							
%						\end{packed_item}
%					\end{packed_item}
				
					
				\subsubsection{Dijagrami obrazaca uporabe}
					
					\textit{Prikazati odnos aktora i obrazaca uporabe odgovarajućim UML dijagramom. Nije nužno nacrtati sve na jednom dijagramu. Modelirati po razinama apstrakcije i skupovima srodnih funkcionalnosti.}
				\eject		
				
			\subsection{Sekvencijski dijagrami}
				
				\textbf{\textit{dio 1. revizije}}\\
				
				\textit{Nacrtati sekvencijske dijagrame koji modeliraju najvažnije dijelove sustava (max. 4 dijagrama). Ukoliko postoji nedoumica oko odabira, razjasniti s asistentom. Uz svaki dijagram napisati detaljni opis dijagrama.}
				\eject
	
		\section{Ostali zahtjevi}
		
			\textbf{\textit{dio 1. revizije}}\\
		 
			 \textit{Nefunkcionalni zahtjevi i zahtjevi domene primjene dopunjuju funkcionalne zahtjeve. Oni opisuju \textbf{kako se sustav treba ponašati} i koja \textbf{ograničenja} treba poštivati (performanse, korisničko iskustvo, pouzdanost, standardi kvalitete, sigurnost...). Primjeri takvih zahtjeva u Vašem projektu mogu biti: podržani jezici korisničkog sučelja, vrijeme odziva, najveći mogući podržani broj korisnika, podržane web/mobilne platforme, razina zaštite (protokoli komunikacije, kriptiranje...)... Svaki takav zahtjev potrebno je navesti u jednoj ili dvije rečenice.}
			 
			 
			 
	